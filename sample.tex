\documentclass[a4paper,10.5pt]{jsarticle}


% 数式
\usepackage{amsmath,amsfonts}
\usepackage{bm}

% 国際単位系
\usepackage{siunitx}

% 画像
\usepackage[dvipdfmx]{graphicx}

%回路図
\usepackage{amsmath,amssymb}
\usepackage{siunitx}
\usepackage{float}
\usepackage{tikz}
\usepackage[european]{circuitikz}


\ctikzset{stroke diode}
\ctikzset{led arrows from cathode}

\begin{document}


\section{目的}
\begin{itemize}
  \item かじょうがき1
  \item かじょうがき2
  \item かじょうがき3
\end{itemize}

\section{実験装置}
\begin{itemize}
  \item かじょうがき1
  \item かじょうがき2
\end{itemize}

\section{実験方法}
\subsection{実験方法の関係}
\subsubsection{実験方法の関係1}
じっけんほうほう
\subsubsection{実験方法の関係2}
図\ref{fig:よいかいろ1}の回路図の参照.単位を付けるとこう.$E=\SI{5.00}{\volt},R=\SI{220}{\kilo\ohm}$とした.
\begin{figure}[H]
  \centering
    \begin{circuitikz}[european]
      \draw(0,4) to[battery1,label=$E$] (0,0)
                to [short](2,0)
                to [short,*-](2,2)
                to [pR,mirror,name=pd] (2,4)--(0,4);
      \draw(2,0) to [short] (3,0)
                  to [R=$R$](3,3)
                  to [short](pd.wiper);
    \end{circuitikz}
  \caption{よいかいろ1}
  \label{fig:よいかいろ1}
  \label{}
\end{figure}

\subsection{実験特性と実験回路}
\subsubsection{各実験特性}
\label{sec:各実験特性}
図\ref{fig:よいかいろ2}の回路図の参照方法.オームとかはこう$R=\SI{1.0}{\kilo\ohm}$だよ.起電力$E$を$\SI{0}{\volt}$から上昇させる.電流はこう,順電流$I_F$を$\SI{10}{\milli\ampere}$だお.添字付きの電圧だお,$V_F$
\begin{figure}[H]
  \centering
  \begin{circuitikz}[straight voltages]
    \draw(0,2) to[battery1,l_=$E$,i=$I_F$] (0,0) -- (4,0)
          (4,2) to [leD,v^<=$V_F$](4,0)
          (0,2) to [R=$R$] (4,2);
  \end{circuitikz}
  
  \caption{よいかいろ2}
  \label{fig:よいかいろ2}
\end{figure}

\subsubsection{各実験回路}
\label{sec:各実験回路}
\ref{sec:各実験特性}セクションの参照だよ.図\ref{fig:よいかいろ2}図の参照だよ.$E=\SI{5.00}{\volt},I_F=\SI{10}{\milli\ampere}$電圧電流だよ.$R$抵抗だよおおおおおおおおおおおおおおおおおおおおおおお.

\subsection{実験の動作}
\subsubsection{実験する入力電圧}
図\ref{fig:よいかいろ3}の図だよ.$E=\SI{5.00}{\volt}$だよ.$R$抵抗だよ.\ref{sec:各実験回路}セクションの参照だよ.くぁwせdrっftgyふじこlp;「’」

だんらく2はこうかくよあqswでfrgthyじゅきぉ;p’「くぁsでfrgthyじゅきぉ;p’「くぁwsでrftgyふじこlp;「’」

\begin{figure}[H]
  \centering
  \begin{circuitikz}[european]
    \draw(0,0) to [battery1,invert,label=$E$] (0,4) ;
    \draw (0,0) -- (3,0) -- (3,0.75);
    \draw(1,4) to [pR, *-*, name=pd] (1,0);
    \draw node[american not port](not) at (3,1) {};
    \draw (pd.wiper) |- (not.in);
    \draw (0,4) -- (4,4)
            to [R=$R$] (4,2)
            to [leD] (4,1)
            to [short] (not.out);
  \end{circuitikz}
  \caption{よいかいろ3}
  \label{fig:よいかいろ3}
\end{figure}
\end{document}